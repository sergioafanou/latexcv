%-----------------------------------------------------------------------------------------------------------------------------------------------%
%	The MIT License (MIT)
%
%	Copyright (c) 2016 Jan Küster
%
%	Permission is hereby granted, free of charge, to any person obtaining a copy
%	of this software and associated documentation files (the "Software"), to deal
%	in the Software without restriction, including without limitation the rights
%	to use, copy, modify, merge, publish, distribute, sublicense, and/or sell
%	copies of the Software, and to permit persons to whom the Software is
%	furnished to do so, subject to the following conditions:
%	
%	THE SOFTWARE IS PROVIDED "AS IS", WITHOUT WARRANTY OF ANY KIND, EXPRESS OR
%	IMPLIED, INCLUDING BUT NOT LIMITED TO THE WARRANTIES OF MERCHANTABILITY,
%	FITNESS FOR A PARTICULAR PURPOSE AND NONINFRINGEMENT. IN NO EVENT SHALL THE
%	AUTHORS OR COPYRIGHT HOLDERS BE LIABLE FOR ANY CLAIM, DAMAGES OR OTHER
%	LIABILITY, WHETHER IN AN ACTION OF CONTRACT, TORT OR OTHERWISE, ARISING FROM,
%	OUT OF OR IN CONNECTION WITH THE SOFTWARE OR THE USE OR OTHER DEALINGS IN
%	THE SOFTWARE.
%
%	*************	RESOURCES USED	 ********************
%
%	http://tex.stackexchange.com/questions/5718/package-for-pie-charts
%	http://tex.stackexchange.com/questions/183087/draw-colored-world-us-map-in-latex#183138
%	http://www.texample.net/tikz/examples/simple-flow-chart/
%	http://vizualize.me/#
%	http://devnet.kentico.com/getattachment/fd511a92-e164-40f5-ba51-07c228a49fed/Kentico_fortune500_infographics_FINAL.png
%
%-----------------------------------------------------------------------------------------------------------------------------------------------%


%============================================================================%
%
%	DOCUMENT DEFINITION
%
%============================================================================%

%we use article class because we want to fully customize the page
\documentclass[10pt,A4]{article}	


%----------------------------------------------------------------------------------------
%	ENCODING
%----------------------------------------------------------------------------------------

%we use utf8 since we want to build from any machine
\usepackage[utf8]{inputenc}		

%----------------------------------------------------------------------------------------
%	LOGIC
%----------------------------------------------------------------------------------------

\usepackage{xifthen}
\usepackage{calc}

%----------------------------------------------------------------------------------------
%	FONT
%----------------------------------------------------------------------------------------

% some tex-live fonts - choose your own

%\usepackage[defaultsans]{droidsans}
%\usepackage[default]{comfortaa}
%\usepackage{cmbright}
\usepackage[default]{raleway}
%\usepackage{fetamont}
%\usepackage[default]{gillius}
%\usepackage[light,math]{iwona}
%\usepackage[thin]{roboto} 
\usepackage{hyperref}
\hypersetup{
    colorlinks=true,
    linkcolor=blue,
    filecolor=magenta,      
    urlcolor=cyan,
}

% set font default
\renewcommand*\familydefault{\sfdefault} 	
\usepackage[T1]{fontenc}

% more font size definitions
\usepackage{moresize}		

% awesome font
\usepackage{fontawesome5}


%----------------------------------------------------------------------------------------
%	PAGE LAYOUT  DEFINITIONS
%----------------------------------------------------------------------------------------

%debug page outer frames
%\usepackage{showframe}			

%define page styles using geometry
\usepackage[a4paper]{geometry}		

% for example, change the margins to 2 inches all round
\geometry{top=1cm, bottom=1cm, left=1cm, right=1cm} 	

% use customized header
\usepackage{fancyhdr}				
\pagestyle{fancy}

%less space between header and content
\setlength{\headheight}{-5pt}		

% customize header entries
\lhead{}
\rhead{}
\chead{}

%indentation is zero
\setlength{\parindent}{0mm}

%----------------------------------------------------------------------------------------
%	TABLE /ARRAY DEFINITIONS
%---------------------------------------------------------------------------------------- 

%extended aligning of tabular cells
\usepackage{array}

% custom column width
\newcolumntype{x}[1]{%
>{\raggedleft\hspace{0pt}}p{#1}}%


%----------------------------------------------------------------------------------------
%	GRAPHICS DEFINITIONS
%---------------------------------------------------------------------------------------- 

\usepackage{graphicx}
\usepackage{wrapfig}

% for drawing graphics and charts
\usepackage{tikz}
\usetikzlibrary{shapes, backgrounds}

% use to vertically center content
% credits to: http://tex.stackexchange.com/questions/7219/how-to-vertically-center-two-images-next-to-each-other
\newcommand{\vcenteredinclude}[1]{\begingroup
\setbox0=\hbox{\includegraphics{#1}}%
\parbox{\wd0}{\box0}\endgroup}

% use to vertically center content
% credits to: http://tex.stackexchange.com/questions/7219/how-to-vertically-center-two-images-next-to-each-other
\newcommand*{\vcenteredhbox}[1]{\begingroup
\setbox0=\hbox{#1}\parbox{\wd0}{\box0}\endgroup}

%----------------------------------------------------------------------------------------
%	ICON-SET EMBEDDING
%---------------------------------------------------------------------------------------- 

% at this point we simplify our icon-embedding by simply referring to a set of png images.
% if you find a good way of including svg without conflicting with other packages you can
% replace this part
\newcommand{\icon}[2]{\colorbox{thirdcol}{\makebox(#2, #2){\textcolor{titletext}{\csname fa#1\endcsname}}}}	%icon shortcut
\newcommand{\icontext}[3]{ 						%icon with text shortcut
	\vcenteredhbox{\icon{#1}{#2}} \vcenteredhbox{\textcolor{textcol}{#3}}
}

\newcommand{\icox}[4]{\colorbox{#3}{\makebox(#2, #2){\textcolor{#4}{\csname fa#1\endcsname}}}}	%icon shortcut
\newcommand{\iconbox}[5]{ 						%icon with text shortcut
	\vcenteredhbox{\icox{#1}{#2}{#3}{#5}} \vcenteredhbox{\textcolor{#5}{\colorbox{#3}{#4}}}
}

\newcommand{\icoxfa}[4]{\colorbox{#3}{\makebox(#2, #2){\textcolor{#4}{\faIcon{#1}}}}}	%icon shortcut
\newcommand{\iconboxfa}[5]{ 						%icon with text shortcut
	\vcenteredhbox{\icoxfa{#1}{#2}{#3}{#5}} \vcenteredhbox{\textcolor{#5}{\colorbox{#3}{#4}}}
}

\newcommand{\iconref}[2]{\textcolor{#1}{\csname fa#2\endcsname}}	%icon shortcut
\newcommand{\iconweb}[5]{ 						%icon with text ref shortcut
	\href{https://www.#5}{\vcenteredhbox{\iconref{#1}{#2}} \textcolor{#3}{#4}}
}

%----------------------------------------------------------------------------------------
%	Color DEFINITIONS
%---------------------------------------------------------------------------------------- 

\usepackage{xcolor}

%defineColors
\definecolor{pyblue}{HTML}{4B8BBE}
% \definecolor{pyblue}{HTML}{306998}

\definecolor{lpyellow}{HTML}{FFE873}
\definecolor{pyellow}{HTML}{FFD43B}

\definecolor{pygray}{HTML}{646464}

\definecolor{sorange}{RGB}{255,150,0}
\definecolor{lblue}{RGB}{0,178,255}
\definecolor{darkblue}{RGB}{0,80,130}
\definecolor{darkerblue}{RGB}{0,100,160}
\definecolor{lgray}{RGB}{0,120,200}
\definecolor{powderblue}{RGB}{190,220,255}
\definecolor{darkestblue}{RGB}{0,50,80}


%main color
\colorlet{maincol}{pyellow}

%secondary colors
%\colorlet{secondcol}{lblue}
\colorlet{secondcol}{pygray}
\colorlet{thirdcol}{darkblue}
\colorlet{fourthcol}{pyblue}
\colorlet{fifthcol}{lgray}
\colorlet{sixthcol}{darkblue}

%background color
\colorlet{bgcol}{powderblue}

%textcolor
\colorlet{textcol}{darkestblue}

%titletextcolor
\colorlet{titletext}{white}

%sectioncolor
\colorlet{sectcol}{white}

%set a background col for whole page
\pagecolor{bgcol}


%----------------------------------------------------------------------------------------
% 	HEADER
%----------------------------------------------------------------------------------------

% remove top header line
\renewcommand{\headrulewidth}{0pt} 

%remove botttom header line
\renewcommand{\footrulewidth}{0pt}	  	

%remove pagenum
\renewcommand{\thepage}{}	

%remove section num		
\renewcommand{\thesection}{}			


%----------------------------------------------------------------------------------------
%
% 	TIKZ GRAPHICS
%
%----------------------------------------------------------------------------------------


% the chart graphics are outsourced into own files

%----------------------------------------------------------------------------------------
% 	PIE CHART
%----------------------------------------------------------------------------------------
\input{../g/piechart.tex}

%----------------------------------------------------------------------------------------
% 	BAR CHART
%----------------------------------------------------------------------------------------
\input{../g/barchart.tex}


%----------------------------------------------------------------------------------------
% 	BUBBLE CHART
%----------------------------------------------------------------------------------------
\input{../g/bubbles.tex}

%----------------------------------------------------------------------------------------
% 	SQUARE CHART
%----------------------------------------------------------------------------------------
\input{../g/squares.tex}


%----------------------------------------------------------------------------------------
% 	TIMELINE VERTICAL & HORIZONTAL CHART
%----------------------------------------------------------------------------------------
\input{../g/timeline.tex}

%----------------------------------------------------------------------------------------
% 	FACT BUBBLE
%----------------------------------------------------------------------------------------
\input{../g/factbubble.tex}


%----------------------------------------------------------------------------------------
%	custom sections
%----------------------------------------------------------------------------------------

% create a coloured box with arrow and title as cv section headline
% param 1: section title
%
\newcommand{\cvsection}[2] {
\textcolor{titletext}{\uppercase{\textbf{#1}}}
}

\newcommand{\cvsect}[4]{
	\textcolor{#3}{\hrule}
	\colorbox{#3}{ {\cvsection{#1}{#4}}}
}

% create a coloured arrow with title as cv meta section section
% param 1: meta section title
%
\newcommand{\metasection}[2] {
	\begin{tabular*}{1\textwidth}{ l l }
		#1&#2\\[12pt]
	\end{tabular*}
}

%----------------------------------------------------------------------------------------
%	 CV EVENT
%----------------------------------------------------------------------------------------

% creates a stretched box as 
\newcommand{\cveventmeta}[2] {
	\mbox{\mystrut \hspace{87pt}\textit{#1}}\\
	#2
}

%----------------------------------------------------------------------------------------
% STRUTS AND RULES
%----------------------------------------- -----------------------------------------------

% custom strut
\newcommand{\mystrut}{\rule[-.3\baselineskip]{0pt}{\baselineskip}}

% colored rule and text for chart legends, wrapped in parbox
% param 1: text
% param 2: width in cm or pt, em ...
% param 3: color
\newcommand{\legend}[3]{\parbox[t]{#2}{\textcolor{#3}{\rule{#2}{4pt}}\\#1}}

%----------------------------------------------------------------------------------------
% CUSTOM LOREM IPSUM
%----------------------------------------------------------------------------------------
\newcommand{\lorem}{Lorem ipsum dolor sit amet, consectetur adipiscing elit. Donec a diam lectus.}


%============================================================================%
%
%
%
%	DOCUMENT CONTENT
%
%
%
%============================================================================%
\begin{document}


%use our custom fancy header definitions
\pagestyle{fancy}	


%----------------------------------------------------------------------------------------
%	TITLE HEADLINE
%----------------------------------------------------------------------------------------
\mystrut
\vspace{-12pt}

\hspace{-20pt}\begin{tabular*}{1\textwidth}{ c c c}
	\parbox[c]{0.43\linewidth}{
		\colorbox{thirdcol}{\HUGE{\textcolor{titletext}{\textbf{\uppercase{Sergio Afanou}}}}}\\
		\Large{\textcolor{thirdcol}{\textsc{DevOps engineer & Cloud Expert}}}\\
	}&
	\parbox{0.22\textwidth}{
	\icontext{BirthdayCake}{12}{Né le:  24/07/1994}\\
	\icontext{MapMarker}{12}{Casablanca, Maroc}\\
	\href{callto:+212622902221}{\icontext{Mobile}{12}{+212 6 22 90 22 21}}\\
	\icontext{Car}{12}{Permis : B}\\
	}&
	\parbox{0.4\textwidth}{
	\href{mailto:najmi.achraf@gmail.com}{\icontext{At}{12}{najmi.achraf@gmail.com}}\\
	\href{https://www.github.com/NajmiAchraf}{\icontext{Github}{12}{github.com/NajmiAchraf}}\\
	\href{https://www.stackoverflow.com/users/12854948}{\icontext{StackOverflow}{12}{stackoverflow.com/users/12854948}}\\
	\href{https://www.linkedin.com/in/najmiachraf}{\icontext{Linkedin}{12}{linkedin.com/in/najmiachraf}}\\
	}
\end{tabular*}

% manage space by reducing font size
\small

\vspace{-5pt}

\hspace{-14pt}\begin{minipage}{0.59\textwidth}

%----------------------------------------------------------------------------------------
%	FACTS
%----------------------------------------------------------------------------------------

	\mbox{
		\parbox[c][3cm][c]{0.29\textwidth}{
			\begin{center}
			\textcolor{textcol}{Je suis diplômé en Physique Électronique avec une expérience de travail dans des projets de recherche.}
			\end{center}
		}
		\hspace{2pt}
		\parbox[c][3cm][c]{0.36\textwidth}{
			\begin{center}
			\factbubble{\huge{\textcolor{titletext}{\textbf{LEF}}}\\\small{\textcolor{titletext}{\textbf{Physique Électronique}}}}{1}{maincol}{titletext}{thirdcol}\\
			\textcolor{textcol}{comme dernier diplôme de}\\
			\textcolor{textcol}{\textbf{Université Hassan II de Casablanca : Faculté des Sciences Ben M'sik.}}
			\end{center}
		}
		\hspace{2pt}
		\parbox[c][3cm][c]{0.29\textwidth}{
			\begin{center}
			\textcolor{textcol}{Je suis toujours à la recherche de nouveaux projets interdisciplinaires passionnants liés à la physique.}
			\end{center}
		}
	}
	\vspace{20pt}

%----------------------------------------------------------------------------------------
%	COMPETENCES
%----------------------------------------------------------------------------------------

	\cvsect{Competences}{0.49}{thirdcol}{textcol}\\
	\mbox{
		
		% TEXT BOX
		\parbox[b][130pt][c]{0.35\textwidth}{
			\begin{center}
			\textcolor{textcol}{La plus grande partie de ma contribution à mon travail se situe dans le domaine du développement et maintenance informatique.}
			\end{center}
			\vspace{-10pt}
			% LANGUAGES
			\cvsect{Langues}{0.49}{thirdcol}{textcol}\\[4pt]
			\iconbox{Language}{12}{thirdcol}{Français (B2)}{titletext}\\
			\iconbox{Language}{12}{thirdcol}{Anglais (A2)}{titletext}\\
		}

		% PIE CHART	
		\parbox[b][130pt][c]{0.65\textwidth}{
			\footnotesize
			\begin{piechart}{360}{1.7}{bgcol}{textcol}{sectcol}
				\slice{33}{Développement}{maincol}
				\slice{27}{Maintenance}{thirdcol}
				\slice{23}{Schématisation}{secondcol}
				\slice{17}{Désigne}{fifthcol}
			\end{piechart}\\
		}
	}
	% \begin{center}
	% \vspace{-10pt}
	% \begin{tikzpicture}
	% 	\draw[draw=titletext,dashed, opacity=0.5] (4,0) -- (-4,0);
	% \end{tikzpicture}
	% \end{center}
	
	% \mbox{
		% \begin{timelinehorizontal}{2019}{2022}{9}{textcol}{0.5}
		
			% \cvevent{3/2021}{3/2021}{Developer Certification • Scientific Computing with Python}{2}{-2.5}{maincol}{-1}
	
			% \cvevent{3/2021}{9/2019}{\href{https://freecodecamp.org/certification/najmiachraf/scientific-computing-with-python-v7}{\textcolor{textcol}{\faFreeCodeCamp} \hspace{1pt} freecodecamp.org/certification/najmiachraf}}{1.5}{-3}{maincol}{-1}
	
			% \cvevent{3/2021}{9/2019}{\href{https://freecodecamp.org/certification/najmiachraf/scientific-computing-with-python-v7}{/scientific-computing-with-python-v7}}{1}{-2}{maincol}{-1}
	
			% \cvevent{9/2019}{3/2021}{Beginning of learning Python}{0.5}{-5}{maincol}{-1}
	
			% \cvevent{6/2019}{9/2017}{Bachelor of General Studies (BGS) • Physics Electronic}{1.5}{-5}{maincol}{-1}
	
			% \cvevent{6/2017}{3/2017}{Associate of General Studies (AGS) •  Physics}{1}{-8}{maincol}{-1}
	
		% \end{timelinehorizontal}
	% }

%----------------------------------------------------------------------------------------
%	PROGRAMMES LOGICIELS
%----------------------------------------------------------------------------------------

	% \vspace{-10pt}
	\cvsect{Software Programs}{0.49}{thirdcol}{textcol}\\[-15pt]
	\begin{center}
	\mbox{
		\bubbles{6/PyCharm/\faIcon{file-code}, 4.5/MD/\faMarkdown, 5.5/CLion/\faIcon{file-code}, 4/Terminal/\faTerminal, 5/Proteus/\faDiceD6, 3.5/\faGit/\faCodeBranch}{\cvsection{Utilitaires}}
	}
	\vspace{5pt}
	\newline
	\parbox[b][40pt][c]{0.35\textwidth}{
		% OS
		\cvsect{OS}{0.35}{thirdcol}{textcol}\\[4pt]
		\iconbox{Microsoft}{12}{thirdcol}{Windows}{titletext}\\
		\iconbox{Apple}{12}{thirdcol}{macOS}{titletext}\\
		}
	\hspace{10pt}
	\parbox[b][50pt][c]{0.6\textwidth}{
		\begin{center}
		\textcolor{textcol}{J'utilise régulièrement pour mes projets de développement PyCharm et CLion comme IDE, et j'utilise pour la capture schématique et la simulation Proteus CAD.}
		\end{center}
		}
		
	\end{center}
		
%----------------------------------------------------------------------------------------
%	DÉVELOPPEMENT INFORMATIQUE
%----------------------------------------------------------------------------------------

	\vspace{10pt}
	% TEXT BOX
	\parbox[b][25pt][c]{1\textwidth}{
		% DÉVELOPPEMENT INFORMATIQUE
		\cvsect{Developpement Informatique}{0.49}{thirdcol}{textcol}\\[4pt]
		\iconbox{Python}{12}{maincol}{Python 3}{textcol}
		\iconbox{Cuttlefish}{12}{maincol}{C}{textcol}
		\hspace{20pt}
		\legend{Langage}{1.5cm}{maincol} \legend{Front End}{1.5cm}{secondcol} \legend{Back End}{1.5cm}{fifthcol}
	}
	
	% BAR CHART
	\vspace{9pt}\mbox{\hspace{-14pt}
		\begin{barchart}{10}{5.2}{sectcol}{textcol}{sectcol}{maincol}{secondcol}{fifthcol}
% n1*x + n2*x + n3*x... = 100, Ex:{x=0.459770114942529, n1=80} -> x*n1 = 36.7816
			% \baritem{50}{\faPython \hspace{1pt} Python 3}{36.7816}{33.3333}{29.885}
			\baritem{50}{\faCode \hspace{1pt} Syntaxe}{100}{0}{0}
			\baritem{50}{\faBoxes \hspace{1pt} Bibliothèques}{0}{0}{90}
			\baritem{50}{\faWaveSquare \hspace{1pt} Data Science}{0}{40}{40}
			\baritem{50}{\faLaptopCode \hspace{1pt} GUI}{0}{70}{0}
			\baritem{50}{\faChartPie \hspace{1pt} Data Analyse}{0}{0}{60}
			
		\end{barchart}
		
		\hspace{15pt}
		
		% TEXTBOX
		\parbox[b][70pt][c]{0.22\textwidth}{\vspace{30pt}
		\colorbox{maincol}{\faPython \hspace{1pt} \& \faCuttlefish \hspace{1pt} \& \LaTeX}	
		
		\iconboxfa{share-alt}{8}{fifthcol}{intégré}{titletext}
		
		\iconbox{FileCode}{8}{fifthcol}{pyinstaller}{titletext}
		
		\iconbox{ChartArea}{8}{secondcol}{matplotlib}{titletext}
		
		\iconboxfa{square-root-alt}{8}{fifthcol}{sympy}{titletext}
		
		\iconboxfa{feather-alt}{8}{secondcol}{tkinter}{titletext}
		
		\iconbox{SortNumericDown}{8}{fifthcol}{numpy}{titletext}
		
		\iconbox{Table}{8}{fifthcol}{pandas}{titletext}
		}
	}\\
	
	% TEXTBOX
	\parbox[b][35pt][c]{0.73\textwidth}{
		\begin{center}
		\textcolor{textcol}{Actuellement, je développe des applications en temps réel avec Python, et crée une interface graphique avec tkinter, et extrait du script `file.py` un fichier exécutable par pyinstaller en utilisant auto py to exe.}
		\end{center}
		}
	% \begin{center}
	% \begin{tikzpicture}
	% 	\draw[draw=titletext,dashed, opacity=0.5] (4,0) -- (-4,0);
	% \end{tikzpicture}
	% \end{center}
	% \vspace{-10pt}
	\vspace{5pt}
	
%---------------------------------------------------------------------------------------
%	ACTIVITIES
%----------------------------------------------------------------------------------------
	\cvsect{Activites}{0.49}{thirdcol}{textcol}\\[14pt]
	\mbox{
		
		% SQUARE BARS
		\parbox[b][3cm][c]{4.6cm}{
			\squares{24/\faRunning \hspace{0.5pt} Sport,27/\faChess \hspace{1pt} Jeux Cérébrale,30/\faNewspaper \hspace{1pt} Actualités Tech,33/\faPlayCircle \hspace{1pt} Tutoriels}{1}
			}
		
		% FACT BUBBLE
		\parbox[b][3cm][c]{3cm}{
			\begin{center}
			\factbubble{\HUGE{\textcolor{titletext}{\textbf{8}}}\\\small{\textcolor{titletext}{\textbf{repos \\ publique}}}}{0.85}{maincol}{titletext}{thirdcol}\\
			\textcolor{textcol}{dans mon}\\
			\textcolor{textcol}{\textbf{Compte GitHub \\ \faGithub}}
			\end{center}
		}
		
		% TEXT BOX
		\parbox[b][3cm][c]{3.5cm}{
			\begin{center}
			\textcolor{textcol}{Parmis les dominantes activités que je pratique est de naviguer sur internet à la recherche de nouvelles sur le monde d'informatique et  entraîné sur tous les tutoriels qui m'intéresse.}
			\end{center}
		}

	}

\end{minipage}
\hspace{5pt}
\begin{minipage}{0.01\textwidth}
	\begin{center}
		\begin{tikzpicture}
			\draw[draw=titletext,dashed, opacity=0.5] (0,-12) -- (0,12);
		\end{tikzpicture}
	\end{center}
\end{minipage}
\begin{minipage}{0.43\textwidth}
%----------------------------------------------------------------------------------------
%	EDUCATION / PROJET / EXPERIENCE
%----------------------------------------------------------------------------------------
\cvsect{Education et Projet et Experience}{0.41}{thirdcol}{textcol}\\% [16pt]


%----------------------------------------------------------------------------------------
%	Physique Électronique
%----------------------------------------------------------------------------------------
\hspace{5pt}\colorbox{maincol}{\footnotesize \faDigitalTachograph \hspace{1pt} Électronique ~•~ \faChargingStation \hspace{1pt} Électrotechnique ~•~ \faMicrochip \hspace{1pt} Numérique}

\hspace{55pt}\colorbox{maincol}{\footnotesize \faBroadcastTower \hspace{1pt} Télécommunication ~•~ \faIcon{network-wired} \hspace{1pt} Réseau}\\
   
\hspace{55pt}\mbox{\legend{Éducation}{1.5cm}{maincol} \legend{Projet}{0.9cm}{secondcol} \legend{Expérience}{1.7cm}{thirdcol} \legend{Événement}{1.7cm}{fifthcol}}

\vspace{-55pt}
\begin{center}

% TIMELINE
\begin{timelinevertical}{2019}{2022}{21}{\linewidth}{0.3}

% {maincol}  {secondcol}  {thirdcol}

% \cvevent{6/2012}{6/2012}{Diplôme du Baccalauréat \newline Série : Sciences Expérimentales. \newline Option : Science de la Vie et la Terre.}{where}{what}}{maincol}{0}

% \cvevent{9/2012}{9/2012}{Début des études Universitaires}{0.5}{0}{maincol}{0}

% \cvevent{2/2017}{9/2018}{Maintenance Informatique}{0.5}{0}{thirdcol}{0.4}

% \cvevent{6/2017}{12/2016}{Diplôme d'Études Universitaires Générales (DEUG) \newline • Filière : Science Matière Physique}{0.5}{0}{maincol}{0.3}

% \cvevent{9/2017}{6/2019}{Début des études de licence}{0.5}{0}{maincol}{0.3}

% \cvevent{2/2018}{6/2018}{Début du Projet de Fin d'Études \newline • Conception d'un système photovoltaïque et réalisation d'un suiveur solaire}{0.5}{0}{secondcol}{0.2}

% \cvevent{6/2018}{6/2018}{Présentation du Projet de Fin d'Études}{0.5}{0}{fifthcol}{0.2}

\cvevent{6/2019}{12/2018}{Licence d'Études Fondamentales (LEF) \newline • Filière : Science Matière Physique \newline • Parcours : Électronique}{0.5}{-0.15}{maincol}{0.3}

\cvevent{9/2019}{8/2021}{Débuter à apprendre Python}{0.5}{0}{maincol}{0.3}

\cvevent{12/2019}{3/2021}{Début du Projet MathPy \newline • Calculatrice Scientifique Moderne}{0.5}{0}{secondcol}{0.2}

\cvevent{4/2020}{3/2021}{Mis le projet MathPy dans GitHub \newline \iconweb{pygray}{Github}{pygray}{github.com/NajmiAchraf/MathPy \newline (PROJET PRIVÉ)}{github.com/NajmiAchraf/MathPy} \newline Projet MathPy est constamment mis à jour}{0.5}{0}{fifthcol}{0.1}

\cvevent{9/2020}{10/2020}{Début du Projet Gmail \newline • Expéditeur CSV de Google messagerie \newline \iconweb{pygray}{Github}{pygray}{github.com/NajmiAchraf/Gmail}{github.com/NajmiAchraf/Gmail}}{0.5}{0}{secondcol}{0.2}

\cvevent{2/2021}{3/2021}{Début du Projet Hydrogéologie \newline • Hydrologie des eaux souterraines \newline \iconweb{pygray}{Github}{pygray}{github.com/NajmiAchraf \newline /Hydrogeologie (PROJET PRIVÉ)}{github.com/NajmiAchraf/Hydrogeologie}}{0.5}{-0.7}{secondcol}{0.2}

\cvevent{3/2021}{3/2021}{Scientific Computing with Python \newline Developer Certification \newline \iconweb{pygray}{FreeCodeCamp}{pygray}{freecodecamp.org/certification/\newline najmiachraf/scientific-computing-with-python-v7}{freecodecamp.org/certification/najmiachraf/scientific-computing-with-python-v7}}{0.5}{1}{maincol}{0.3}

\cvevent{7/2021}{8/2021}{Un mois intensif dans UM6P \newline Piscine d'école 1337 42 network \newline \iconweb{pygray}{Github}{pygray}{github.com/NajmiAchraf/1337\_Piscine}{github.com/NajmiAchraf/1337\_Piscine}}{0.5}{1}{maincol}{0.3}

\end{timelinevertical}
\end{center}
\vspace{-40pt}\end{minipage}
%============================================================================%
%
%
%
%	DOCUMENT END
%
%
%
%============================================================================%
\end{document}
